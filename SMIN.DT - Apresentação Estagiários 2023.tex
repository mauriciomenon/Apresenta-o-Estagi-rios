\documentclass{beamer}

\usepackage[utf8]{inputenc}
\usepackage[T1]{fontenc}
\usepackage[english,portuguese]{babel}
\usepackage{graphicx}
\usepackage{hyperref}

% Tema escuro
\usetheme{Warsaw}
\usecolortheme{crane} 

% Ajustes de cores
\setbeamercolor{background canvas}{bg=black}
\setbeamercolor{frametitle}{fg=white,bg=darkgray}
\setbeamercolor{title}{fg=yellow,bg=black}
\setbeamercolor{itemize item}{fg=white}
\setbeamercolor{itemize subitem}{fg=white}
\setbeamercolor{normal text}{fg=white}
\setbeamercolor{section in toc}{fg=white} 
\setbeamercolor{subsection in toc}{fg=white}

\setbeamertemplate{headline}{%
  \leavevmode%
  \hbox{%
    \begin{beamercolorbox}[wd=\paperwidth,ht=2.5ex,dp=1.125ex]{palette quaternary}%
    \insertsectionnavigationhorizontal{\paperwidth}{}{\hskip0pt plus1filll}
    \end{beamercolorbox}%
  }
}

\title{\textbf{Engenharia de Manutenção Eletrônica\\}}
\subtitle{Itaipu Binacional \\ Superintendência de Manutenção \\ Departamento de Engenharia de Manutenção}
\author{SMIN.DT}
\author{SMIN.DT}
\date{14/08/2023}

\begin{document}

\frame{\titlepage}

% Slide de Sumário
\begin{frame}
\frametitle{Sumário e Áreas da SMIN.DT}
\tableofcontents
\end{frame}

\section[Introdução]{Introdução}
\begin{frame}
\frametitle{Introdução}
\begin{itemize}
    \item Análise da confiabilidade e disponibilidade de ativos da AI;
    \item Manter a base de dados dos ativos atualizada;
    \item Fornecer subsídios para a Man. Periódica de equipamentos;
    \item Realizar análise de desempenho de sistemas;
    \item Atuar como suporte de outras áreas em Man. Aperiódicas;
    \item Analisar a necessidade de compra de sobressalentes.
    \end{itemize}
\end{frame}

\begin{frame}
\frametitle{Introdução}
\begin{itemize}
    \item Garantir o funcionamento e sobrevida de sistemas em funcionamento através de atualizações;
    \item Propor a área de Engenharia de Projeto modificações e alterações baseadas nas análises listadas anteriormente;
    \item Gerenciar contratos de suporte e manutenção;
    \item Gerenciar e aplicar políticas de segurança dos dados dos ativos;
    \item Segurança cibernética e monitoramento, estudo e aplicabilidade de normas.
\end{itemize}
\end{frame}

\section[IEE1]{IEE1: Proteção e Controle}
\begin{frame}
\frametitle{IEE1: Proteção e Controle}
\begin{itemize}
    \item Proteção controle e sincronismo:
    \item - 20 unidades geradoras;
    \item - Subestação da GIS; 
    \item - Serviços Auxiliares;
    \item Esquemas de Controle e Emergência;
    \item Subestação da Margem Direita.
\end{itemize}
\end{frame}

\section[IEE2]{IEE2: Regulação}
\begin{frame}
\frametitle{IEE2: Regulação}
\begin{itemize}
    \item Regulação de velocidade e de tensão (excitação) das 20 UGs;
    \item Regulação de velocidade e de tensão de 4 geradores diesel;
    \item Pontes rolantes e pórticos (Comando e Controle);
    \item Registradores de Perturbações e Medição Fasorial;
    \item Sistema de controle da Iluminação Monumental.
\end{itemize}
\end{frame}

\section[IEE3]{IEE3: Automação, Controle e Telecomunicações}
\begin{frame}
\frametitle{IEE3: Automação, Controle e Telecomunicações}
\begin{itemize}
    \item Sistemas de Supervisão e Controle;
    \item SCC: SCADA Siemens (SEMD);
    \item Atualização do SCADA ABB/Hitachi (usina);
    \item Modernização das UTRs (usina);
    \item Infraestrutura de sistemas com servidores, virtualização e SO;
    \item Teleproteção via sistemas SDH;
    \item Dados hidrometeorológicos via rádio/satélite (STH);
    \item Segurança da barragem (ADAS);
    \item Telefonia IP da usina/SEMD e  Telefonia Operativa;
    \item Segurança física da usina (ESAI).
\end{itemize}
\end{frame}

\section[IEE4]{IEE4: Monitoramento, Análise de Dados e Seg. Cibernética}
\begin{frame}
\frametitle{IEE4: Monitoramento, Análise de Dados e Seg. Cibernética}
\begin{itemize}
    \item Segurança Cibernética;
    \item Monitoramento e gestão de ativos digitais da AI;
    \item Apoio a administradores dos sistemas;
    \item Rede industrial (SIRI e futura RTA): Datacenter, redes sem fio, entre outros services;
    \item Manutenção Preditiva;
    \item Ferramentas PI: historiador de dados de processo, implementação de cálculos e painéis de visualização;
\end{itemize}
\end{frame}

\section[Conclusão]{Conclusão}
\begin{frame}
\frametitle{Conclusão}
\begin{itemize}
    \item O compromisso com a segurança, confiabilidade e disponibilidade dos ativos eletrônico/digitais;
    \item A importância da análise dos dados e indicadores;
    \item Manter e atualizar sistemas em ciclos de vida cada mais curtos;
    \item Expandir a manutenção preditiva;
    \item O desafio da Atualização Tecnológica em colaboração com Engenharia de Projeto.
\end{itemize}
\end{frame}

\begin{frame}
\frametitle{Contato}
\centering
Eng. Sr. Maurício Menon \\
Itaipu Binacional\\
SMIN.DT - Engenharia de Manutenção Eletro-Eletrônica\\
IEE3 - Automação, Controle e Telecomunicações\\
\href{mailto:mauricio.menon@unioeste.br}{mauricio.menon@unioeste.br} \\
\href{mailto:menon@itaipu.gov.br}{menon@itaipu.gov.br}
\end{frame}

\end{document}
